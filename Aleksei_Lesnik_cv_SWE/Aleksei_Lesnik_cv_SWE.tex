\documentclass[12pt,a4paper]{moderncv}
\moderncvstyle{classic}
\moderncvcolor{orange}
\usepackage{lipsum}
\usepackage[scale=0.8]{geometry}
\usepackage{geometry}
\geometry{verbose, a4paper, tmargin = 1.5cm, bmargin = 1cm, lmargin = 1cm, rmargin = 1cm}

\begin{document}

    \definecolor{Attention}{RGB}{255, 75, 0} 

	\begin{minipage} {0.7\textwidth}
		\begin{flushleft}
			\huge \textbf{Aleksei Lesnik} 
			\newline
			\textcolor[gray]{0.4} {
			    \indent Software Engineer / Programmer\\
			}
			\normalsize
		\end{flushleft}
	\end{minipage}
	\begin{minipage} {0.3\textwidth}
		\begin{flushright}
			\textcolor[gray]{0.4} {
				\indent Saint-Petersburg, Russia\\
				\indent Mobile: +7(981)107-60-21  \\
				\indent Email : \href{mailto:lesnik-a-a@yandex.ru} {lesnik-a-a@yandex.ru}\\ 
				\indent GitHub : \href{https://github.com/a-alex-l} {a-alex-l}\\  	
			}
		\end{flushright}
	\end{minipage}

    \section{Education}
    
        \cventry{2019 -- 2023}
            {\href {http://english.spbu.ru/education/undergraduate/bachelor/85-program-bac/2496-modern-programming}
                {Bachelor, Modern Programming in SPSU}}
            {}{}
            {Saint-Petersburg}
            {First year: C++ with Code Review, Algorithms, Calculus, Algebra, Discrete Math, UNIX\newline
             Second year: Kotlin, Haskell, Algorithms, Calculus, Linear Algebra, Computer architecture}
        
    \section{GitHub and Projects}
        
        \cventry{Spring 2020}
            {\href {https://github.com/a-alex-l/MapLap} {MapLap}} {Circle and Line finder, 3 participants}
            {\newline MapLap is returning \LaTeX code. You can load a photo, scan or screenshot} {}
            {I developed Detector Algorithm. My tools ware \textcolor{Attention} {Python \& OpenCV} }
        
        \cventry{Summer 2020 \newline (current) }
            {\href{https://github.com/a-alex-l/MapLap_2} {MapLap 2.0}} {Ellipses and Line finder}
            {\newline Highly Optimized Console Image application} {}
            {I made application with test cases and my own Hough Transform. My tools were \textcolor{Attention} {C++ \& OpenCV}}
        
        \cventry{Summer 2020 \newline (current) }
            {\href{https://github.com/a-alex-l/TetrisBuilder} {Tetris Builder}} {Mobile Game, Physics Tetris, 2 participants}
            {\newline A game based on tetris blocks, also blocks can fall and collide} {}
            {I implemented the Game Process. My tool was \textcolor{Attention} {C++ \& Godot Engine} }
        
    \section{Summer and Winter Coding Camps}
        
        \cventry{2019 \& 2018} {Summer Informatics Camp} {}{} {Algorithms and Data Structures} {}
        
        \cventry{2018 \& 2017} {Summer School Olympiad Prep} {}{} {Olympiad Algorithms and Data Structures} {}
        
        \cventry{2016-2019} {ITMO Advanced Programming Section} {}{} {Algorithms and Data Structures} {}
        
        \cventry{} {Summary} {}{} { binary search, DSU, MST, BST, vector geometric, graph algorithms, string algorithms, dynamic programming, quick sorting, hashing, FFT, Gauss algorithm} {}
    
    \section{Olympiads and Competitions}
    
        \cventry{} {Team competition in robotics} {} {} {} {}
    
        \cventry{2017} {Inspire Award out of 25 participants}
            {\href {https://www.spbstu.ru/media/news/achievements/team-school-engineers-future-russia-winner-championship-robotics/} {FIRST Tech Challenge Russia}} {} {} {}
        
        \cventry{2018} {Design Award and Winning Aliance Captain out of 30 participants}
            {\href {https://www.spbstu.ru/media/news/achievements/team-polytechnic-quota-world-championship-robotics-usa/}
            {FIRST FTC}} {} {} {}
            
        \cventry{} {Competetive programming results} {} {} {} {}
        
        \cventry{2018} {\href {https://olymp.ifmo.ru/ru/p/it-test/827} {Top 100 out of 1600 participants}}
            {ITMO Olympiad in Computer Science} {} {} {}
        
        \cventry{2019} {\href {https://olymp.ifmo.ru/ru/p/olymp-it-18-19/1212} {Top 160 out of 2900 participants}}
            {ITMO Olympiad in Computer Science} {} {} {}
        
        \cventry{2019} {\href {http://mos-inf.olimpiada.ru/winners10-11-2019} {Top 11 out of 2000 participants}}
            {Moscow Olympiad in Computer Science} {} {} {}
        
        \cventry{2019} {\href {https://diploma.rsr-olymp.ru/files/rsosh-diplomas-static/compiled-storage-2019/by-code/141299110240/color.pdf} {Top 15 out of 2000 participants}}
            {SPSU Olympiad in Computer Science} {} {} {}
            
        \cventry{2019} {\href {https://diploma.rsr-olymp.ru/files/rsosh-diplomas-static/compiled-storage-2019/by-code/142158979050/color.pdf} {Top 17 out of 520 participants}}
            {Open Olympiad in Cognitive Technology Programming} {} {} {}
        
        \cventry{2018} {\href {https://neerc.ifmo.ru/school/ioip/standings-2019.html} {Top 87 out of more than 1000 participants}}
            {Individual Olympiad in Computer Science} {} {} {}
    
    \section{Technical skills}
    
        \cvitem{Languages} {C++, C, Python, Kotlin, Julia }
        
        \cvitem{Tools} {STL, GitHub, SVN, OpenCV, QT, Creo Parametric, Godot, \LaTeX}
    
    \section{Languages}
    
        \cvitem{English} {Intermediate}
        
\end{document}
